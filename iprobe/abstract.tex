\begin{abstract}
   
Application tracing in production systems requires dynamic and flexible instrumentation mechanisms with low-overhead. 
Tracing tools may be required to be started at anytime, and it can take potentially long time periods to collect enough information, but at the same time should not adversely affect service quality.
Existing user-space code monitoring solutions are either inflexible developer-driven static instrumentation which require manual effort, 
or black-box dynamic instrumentation techniques which are flexible but have high overhead.

To solve this problem, we introduce a new hybrid instrumentation technique for user-space code monitoring called {\em iProbe}, which is flexible and has low overhead. 
iProbe takes a novel 2-stage design, and offloads much of the dynamic instrumentation complexity to an offline compilation stage.
It leverages standard compiler flags to introduce ``place-holders" for hooks in the program executables. 
Then it utilizes an efficient user-space HotPatching mechanism which dynamically instruments the functions to be traced and enables execution of instrumented code in a safe and secure manner. 

We implemented iProbe as a dynamic application profiling framework. 
In its evaluation on micro-benchmarks and SPEC CPU2006 benchmark applications,
the iProbe prototype achieved the instrumentation overhead an order of magnitude lower than existing state-of-the-art dynamic instrumentation tools like SystemTap and DynInst. 
We also built a hardware event profiling tool based on iProbe, and were able to obtain function-level hardware event breakdown on SPEC CPU2006 applications with controlled performance overhead (e.g., under $5\%$).   
%This paper shows the effectiveness of our design via evaluation and mysql which is widely used industry software.

 %Usage of open-source tools in production environment for custom application development is an increasing trend in the industry. 
%These tools range from cloud service oriented architectures (OpenStack), to database(mysql) and webservers(apache httpd) etc. 
%Monitoring and mantainence of such tools is increasingly important.


%Monitoring long running deployed software is a challenging task due to the cost in execution and the storage for logs. 
%Dynamic configuration of the logging (hot-patching) is a crucial feature to control the overhead of a selective necessary period of execution and therefore lowers the monitoring cost.

%This paper presents iProbe, a new facility for dynamic instrumentation of user-level software. 
%iProbe has an elegant and effective design which leverages standard compiler flags, binary rewriting mechanisms, and user-drive macros to introduce ``place-holders" for hooks in the program executable. 
%Our tool features an efficient user-space hot-patching mechanism which dynamically instruments the functions to be traced by directly modifying code segments of a running process in a safe and secure manner. 
%iProbe offloads complexities involved in run-time instrumentation by using a novel 2-stage design, and therby incurs instrumentation overhead an order of magnitude lower than existing state of the art tools(Systemtap/Dtrace). 
%This paper shows the effectiveness of our design and evaluates it with a variety of software including CPU SPEC 2006 benchmarks, and commonly used industry software, mysql. 


  %Existing tools lack methodology to flexibly add and remove tracing from user-space functions
% This paper presents iProbe, a new facility for dynamic instrumentation of user-level C/C++ software.
% iProbe leverages standard GCC compiler flags to introduce ``place-holders" in the program executable,
% and offers an automated instrumentation tool to eliminate user effort in adding probe points.
% When not explicitly enabled, iProbe has zero probe effect.
% It allows for arbitrary instrumentation points at program function level,
% and provides a simple command-line interface to assit target function selection at run time.
% iProbe features a user-space hot-patching mechanism which directly modifies code segments of a running process in a safe and secure manner,
% and therefore incurs instrumentation overhead much lower than kernel trapping based dynamic instrumentation mechanisms.
% Our evaluation tested the iProbe prototype on software including CPU SPEC 2006 benchmarks, mysql and Apache, 
% and iProbe is an order of magnitude faster than existing state of the art mechanisms.




%iProbe or “Intelligent Probe” provides a flexible monitoring platform with extremely light weight instrumentation to capture unified software execution traces of both kernel and user-space events. We introduce a novel run-time instrumentation technique which uses existing compiler flags and binary instrumentation to introduce ``place-holders" in the executable which allow for dynamic instrumentation. Further, we provide an easy command line interface to users to select the function to instrument, and to add a library of instrumentation functions. Our implementation is based on use of hot-patching mechanisms which allow us to directly modify code segments of a running process in a safe and secure manner. iProbe can be used for an automated instrumentation tool, which can be packaged with the application, thereby reducing developer effort in adding logging targets in the user code, and allowing easy change of logging targets at run-time. Our solution has been tested on the CPU SPEC 2006 benchmarks, and large middleware applications such as mysql and apache httpd server, and is a order of magnitude faster than existing state of the art mechanisms.

%Additionally, we use iProbe with kernel event tracers as a dual-space tracing solutions so that we avoid expensive context switches from user-space to kernel-space currently in most systematic instrumentation tools.
%\\ \\
%Our advanced functionality provides
%\\ \\
%We have tested our implementation on large scale native applications such as Mysql, Apache
  
\end{abstract}
