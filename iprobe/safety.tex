\section{Discussion: Safety Checks for iProbe}
\label{sec:safety}

Safety and reliability of the instrumentation technique is a big concern for most run-time instrumentation techniques.
One of the key advantages of iProbe is that because of its hybrid design reliability and correctness issues are handled in a better way inherently.
In this section we discuss how our HotPatch can achieve such properties in details.
%As explained earlier iProbe operates in 3 different modes,(1) compiler-assisted (2) User-macro based, and (3) static binary rewriting.

\indent \textbf{HotPatch check against Illegal instructions}: \quad
Unlike previous techniques iProbe relies on compiler correctness to ensure safety and reliability in its binary mode.
%The resulting instrumented binary in ColdPatch is not only safe but optimized.
To ensure correctness in our ColdPatching phase, we convert call instructions to instrumentation functions with \texttt{NOP} instruction. 
This does not in any way effect the correctness of the binary, except that instrumentation calls are not made. 
To ensure run-time correctness, iProbe uses a safety check when it interrupts the application while HotPatching. 
Our safety check pass ensures that the program counters of all threads belonging to the target applications do not point to the region of code that is being overwritten (i.e. \texttt{NOP} instructions are not overwritten while they are being executed.
This check is similar to those from traditional Ptrace\cite{ptrace} driven debuggers etc \cite{kaho,livepatch}. 
Here we use the Ptrace \texttt{GETREGS()} call to inspect the program counter, and if it is currently pointing to the target \texttt{NOP} instructions, we allow the execution to move forward before applying the HotPatch. 
Unlike existing trampoline oriented mechanisms iProbe has a small \texttt{NOP} code segments equal to the length of a single call instruction that it overwrites with instrumentation calls, this means that the check can be performed in a fast and efficient manner. 
It is also important to have this check for all threads which share the code-segment, hence the checking must be able to access the process memory map information, and interrupt all the relevant threads.

\indent \textbf{Safe parameter passing to maintain stack consistency}: \quad
An important aspect for instrumentation is the information passed along to the instrumentation function via the parameter values. 
Since the instrumentation calls are defined by the compiler driven instrumentation, the mechanism in which the parameters passed are decided based on the calling convention used by the compiler. \\
Calling conventions can be broadly classified in two types: caller clean-up based, and callee clean-up based. 
In the former the caller is responsible to pop the parameters passed to function, and hence all parameter related stack operations are performed before and after the call instruction inside the code segment of the caller.
In the later however, the callee is responsible to pop the parameters passed to it.
Since parameters are generally passed using the stack it is important to remove them properly to mantain stack consistency. 

To ensure this iProbe enforces that all calls that are made by the static compiler instrumentation must be \emph{cdecl}~\cite{cdecl} calls where the caller performs the cleanup as compared to \emph{std} calls, where the callee performs it. 
%\emph{cdecl} calls push instrumentation parameters before the call, and stack cleanup is also performed by caller function. 
%On the other hand in \emph{std} call convention the stack cleanup is performed by the callee. 
%For simplicity sake, iProbe uses only \emph{cdecl} calls and allows the push and pop operations performed by the caller. 
As the stack cleanup is automatically performed, it maintains stack consistency, and there is a negligible impact in performance due to the redundant stack operations. 
Alternatively for \emph{std} call convention, push instructions could also be converted to \texttt{NOP}s and HotPatched at run-time, we do not do so as a design choice. 

\indent \textbf{Address Space Layout Randomization}: \quad
Another issue that iProbe addresses is ASLR (address space layout randomization), a security measure used in most environments which randomizes the loading address of executables and shared libraries. 
However, since iProbe assumes the full access to the target system, the load addresses are easily available. 
HotPatcher uses the process id of the target to find all load addresses of each binary/shared library and uses them as base offsets to generate correct instruction pointer addresses. 


%\textbf{Developer Driven Macros}
% The developer driven Macros

%\textbf{Static Binary Rewriter}

