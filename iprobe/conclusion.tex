\section{Summary}
\label{sec:iProbeSummary}

Flexibility and performance have been two conflicting goals for the design of dynamic instrumentation tools.
iProbe offers a solution to this problem by using a two-stage process that offloads much of the complexity involved in run-time instrumentation to an offline stage. 
It provides a dynamic application profiling framework to allow for easy and pervasive instrumentation of application functions
and selective activation. 
We presented in the evaluation that iProbe is significantly faster than existing state-of-the-art tools, and scales well in large application software.

As stated earlier \iprobe is still limited in the sense that the overhead of \iprobe depends on the amount of instrumentation and the instrumentation points.
Similar to other monitoring and instrumentation tools, this makes it impossible to use for higher granularity monitoring or debugging scenarios which can potentially impact production services. 
The results of our experiments with \iprobe motivated us to create \parikshan, which instead of simply reducing the instrumentation overhead, de-couples the instrumentation problem from the user-facing production service.
Instrumentation in \parikshan's debug container has no impact on the production container, and allows debuggers to instrument with higher overheads.

\iprobe can still be used as a standalone debugging tool as well as within the \debugcontainer in \parikshan for assisting debuggers to do execution tracing and thereby catching the error.
It's low overhead can help in an increased and longer \debugwindow as compared to other instrumentation tools.

%It is applicable to both user and kernel event tracing. It can be utilized using 
%a simple GUI interface which reads all instrumentable functions in a binary. 
%iProbe provides for packaging a monitoring system to be deployed with the target application.
%It provides a system to inves tigate performance pathologies, or reliability 
%problems in target binary applications. It has been tested on Linux 32 and 64bit 
%applications.

