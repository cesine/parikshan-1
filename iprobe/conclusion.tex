\section{Conclusion}
\label{sec:conclusion}

Flexibility and performance have been two conflicting goals 
for the design of dynamic instrumentation tools.
iProbe offers a solution to this problem by using a two-stage process that offloads much of the complexity involved in run-time instrumentation to an offline stage. 
It provides a dynamic application profiling framework
to allow for easy and pervasive instrumentation of application functions
and selective activation. 
We presented in the evaluation that iProbe is significantly faster than existing state-of-the-art tools, and scales well in large application software.

%It is applicable to both user and kernel event tracing. It can be utilized using 
%a simple GUI interface which reads all instrumentable functions in a binary. 
%iProbe provides for packaging a monitoring system to be deployed with the target application.
%It provides a system to inves tigate performance pathologies, or reliability 
%problems in target binary applications. It has been tested on Linux 32 and 64bit 
%applications.

%\textbf{Acknowldgements}
