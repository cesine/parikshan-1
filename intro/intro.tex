\chapter{Introduction}
\label{sec:intro}

%Motivation: What is the problem we are trying to solve? Why does it exist? Why is it important to the community?
Although software bugs are nothing new, the complexities of virtualized environments coupled with large distributed systems have made bug localization harder.
The large size of distributed systems means that any downtime has significant financial penalties for all parties involved.
Hence, it is increasingly important to localize and fix bugs in a very short period of time.

%What is the specific problem considered in this work? This paragraph narrows down the topic area of the work. In the first paragraph you have established general context and importance. Here you establish specific context and background.

%In this work, we show that ...". This is the key paragraph in the intro - you summarize, in one paragraph, what are the main contributions of your work given the context you have established in paragraphs 1 and 2. What is the general approach taken? Why are the specific results significant? This paragraph must be really really good. If you can't "sell" your work at a high level in a paragraph in the intro, then you are in trouble. As a reader or reviewer, this is the paragraph that I always look for, and read very carefully.
%You should think about how to structure this one or two paragraph summary of what your work is all about. If there are two or three main results, then you might consider itemizing them with bullets or in test (e.g., "First, ..."). If the results fall broadly into two categories, you can bring out that distinction here. For example, "Our results are both theoretical and applied in nature. (two sentences follow, one each on theory and application)"

%At a high level what are the differences in what you are doing, and what others have done? Keep this at a high level, you can refer to a future section where specific details and differences will be given. But it is important for the reader to know at a high level, what is new about this work compared to other work in the area.

%The remainder of this work is structured as follows..." Give the reader a roadmap for the rest of the work. Avoid redundant phrasing, "In Section 2, In section 3, ... In Section 4, ... " etc.



%Why do existing techniques not solve this problem
Existing state-of-art techniques for monitoring production systems~\cite{dtrace,winetw,systemtap} rely on light-weight dynamic instrumentation to capture execution traces.
Operators then feed these traces to analytic tools~\cite{magpie,clue} to connect logs in these traces and find the root-cause of the error.
However, dynamic instrumentation has a trade-off between granularity of tracing and the performance overhead.
Operators keep instrumentation granularity low, to avoid higher overheads in the production environment.
This often leads to multiple iterations between the debugger and the operator, to increase instrumentation in specific modules, in order to diagnose the root-cause of the bug.
Another body of work has looked into record-and-replay~\cite{odr,revirt,laadan2010transparent,geels2007friday} systems which capture the log of the system, in order to faithfully replay the trace in an offline environment.
Replay systems try and capture system level information, user-input, as well as all possible sources of non-determinism, to allow for in-depth \textit{post-facto} analysis of the error.
However, owing to the amount of instrumentation required, record-and-replay tools deal with an even heavier overhead, making them impractical for real-world production systems.


%Our approach and story - iprobe first then parikshan
The high level goal of this thesis is to present tools and techniques which can help to reduce the time to bug localization, and can be applied in live running production service systems. 
Our initial efforts focused on having the minimum possible instrumentation in the production system, which could at the same time be dynamically turned on or off. 
We developed \iprobe (see chapter~\ref{ch:iprobe}) an intelligent instrumentation tool, which combined the advantages of static instrumentation and dynamic instrumentation to give an order of magnitude better performance in terms of overhead compared to existing state-of-art tools~\cite{dtrace,systemtap,dyninst,pin}.
\iprobe uses placeholders added in the application binary at compile time, which can be leveraged to insert instrumentation when the application is actually running.
In comparison, most current tools use trampoline based techniques (see DTrace~\cite{dtrace}, SystemTap~\cite{systemtap}, Dyninst~\cite{dyninst}), or just in time execution (PIN~\cite{pin}, Valgrind~\cite{valgrind}), requiring complex operations to allow for safe execution and incurs a much higher overhead.
Our compilation driven place-holders allow us to leverage pre-existing space in the binary to safely insert instrumentation and achieve a much better performance.

However, in the process of our experiments we realized one critical limitation of instrumentation based techniques - instrumentation and monitoring is always done within the code, and hence is sequentially executed. 
Since instrumentation will always directly impact the performance of production applications, it needs to be limited to allow for good user experience.
A better way to approach this problem is to ~\emph{decouple debugging instrumentation and application performance}, so that there is no direct impact of the instrumentation on the production application.
This thesis is centered around the idea of a new debugging paradigm called \emph{``live debugging''}, whereby developers can debug/instrument the application while isolating the impact of this instrumentation from the user-facing production application.
The key idea behind this approach is to give faster time-to-bug localization, deeper insight into the health and activity within the system, and to allow operators to dynamically debug applications without fear of changing application behavior.
We leverage existing work in live migration and light-weight user-space container virtualization, to provide an end-to-end workflow for debugging.
Our system replicates the application container into a clone which can be used solely for the purpose of debugging the application.

%observations which led to the design of \parikshan
Our work is inspired by three key observation: 
Firstly, we observe that most service-oriented applications(SOA) are launched on cloud based infrastructures.
These applications use virtualization to share physical resources, maintained by third-party vendors like Amazon EC2~\cite{ec2}, or Google compute~\cite{gcompute} platforms.
Furthermore, there is an increasing trend towards light-weight user-space container virtualization, which is less resource hungry, and makes sharing physical resources easier.
Frameworks like docker~\cite{docker} allow for scaled out application deployment, by allowing each application service instance to be launched in it's own container.
For instance, an application server, and a database server making up a web-service, can be hosted on their own containers, thereby sandboxing each service, and making it easier to scale out.

Secondly, we observe a trend towards Dev-ops~\cite{devops} by the software engineering industry.
DevOps stresses on close coupling between software developers and operators, in order to have shorter release cycles (Facebook web has 2 releases a day, and one mobile release every 4 weeks and Flickr has 10 deployment cycles per day~\cite{releaseFacebookKeynote,10DevOps}).
This re-emphasizes the need to have a very short time to diagnose and fix a bug especially in service oriented application.
We believe by providing a means to observe the application when the bug is active, we will significantly reduce the time to bug localization.

Lastly, our key insight is that for most service-oriented applications (SOA), a failure can be reproduced simply by replaying the network inputs passed on to the application.
For these failures, capturing very low-level sources of non-determinism (e.g. thread scheduling or general system calls, often with high overhead) is unnecessary to successfully and automatically reproduce the buggy execution in a development environment. 
We have evaluated this insight by studying 16 real-world bugs, which we were able to trigger by only duplicating and replaying network packets.
Furthermore we categorized 217 bugs from three real-world applications, finding that most were similar in nature to the 16 that were reproduced, suggesting that our approach would be applicable to them as well.
\xxx{Please look at section so and so for further details regarding the above}

%contributions
\noindent This thesis will make the following contributions:

First, in Chapter~\ref{ch:parikshan} we will present a framework for ``live debugging'' applications while they are running in the production environment.
This will involve a description of our system called \parikshan\footnote{\parikshan is the Sanskrit word for  testing}, which allows \textbf{real-time debugging} without any impact on the production service.
We provide a facility to sandbox the production and debug environments so that any modifications in the debug environment do not impact user-facing operations.
\parikshan avoids the need of large test-clusters, and can target specific sections of a large scale distributed application.
In particular, \parikshan allows debuggers to apply debugging techniques with deeper granularity instrumentation, and profiling without worrying that the instrumentation will impact the production application performance.

In chapter~\ref{ch:NetworkReplaySurvey} we will present details of our case-study presenting real-world bugs which were triggered by network input alone, and which show why using \parikshan would be enough to capture most real-world bugs. 
Each case study presents a different variety of bugs from the following classes: performance, semantic, non-deterministic, configuration and resource leak. 
We believe that these bugs form the most common classification of bugs in service oriented applications.

In chapter~\ref{ch:iprobe} we will present a dynamic instrumentation mechanism called \iprobe.
%, which can assist developers in making applications \parikshan ready.
As explained earlier, chronologically \iprobe was our first tool developed towards achieving the goal of a low-overhead production debugging.
iProbe uses a novel two-stage design, and offloads much of the dynamic instrumentation complexity to an offline compilation stage.
It leverages standard compiler flags to introduce ``place-holders" for hooks in the program executable.
Then it utilizes an efficient user-space ``HotPatching'' mechanism which modifies the functions to be traced and enables execution of instrumented code in a safe and secure manner.
\iprobe can be used as a standalone instrumentation tool or can be used in the \debugcontainer with \parikshan for further assisting the debugger to localize the bug.

In the final chapter~\ref{ch:activedebugging} of this thesis we focus on applications of live debuggging. 
In particular we discuss several existing techniques and how they can be coupled with live debugging. 
We discuss step-by-step scenarios where debugging on the fly can be helpful, and how it can be applied.
We also briefly introduce a new technique called budget limited instrumentation technique for live debugging.
This technique leverages existing work on statistical debugging, and queuing theory to lay a statistical foundation for allocating buffer sizes and various configuration parameters.
It proposes a reactive mechanism to adapt to the overhead of instrumentation bounds using sampling techniques.

%Secondly, we introduce the concept of active debugging.
%This allows for debuggers to add a patch/or a fix in the debug-container and check if it fixes the error. 
%Similarly tests can be performed while ensuring the production container output is not changed, and forward progress is ensured.
%We isolate tests in \debugcontainer to ensure forward progress, and no impact on the production-container.

The rest of this chapter is organized as follows.
Firstly in section~\ref{sec:introDefinition} we define terms and terminologies used in the rest of this thesis.
Section~\ref{sec:introProblemStatement} further defines the scope of our problem statement, definitions, and classifications of the bugs.
Section~\ref{sec:introRequirements} illustrates the requirements this thesis must meet.
Next, in section~\ref{sec:introScope} we define the scope of the techniques presented in this thesis.
Section~\ref{sec:introApproach} briefly goes over the proposed approach presented in this thesis.
In section~\ref{sec:introHypothesis} we give the hypothesis of this thesis.
Section~\ref{sec:introAssumptions} lists some of the assumptions made in this thesis, and section~\ref{sec:introOutline} gives an outline of the organization of the rest of this document.

\section{Definitions}
\label{sec:introDefinition}

Before we further discuss the problem statement, requirements, and approach,this section first formalizes some of the terms used throughout this thesis.

\begin{itemize}
	
	\item \hypertarget{defn:livedebugging}{\textbf{Live Debugging}}
	For the purpose of this thesis, we define \emph{live debugging} as a mechanism to debug applications on the fly while the production services are running and serving end-users.
	
	\item The \hypertarget{defn:development-environment}{\textbf{development environment}} refers to a setting (physical location, group of human developers, development tools, and production and test facilities) in which software is created and tested by software developers and is not made available to end users.
	The debugging process in the development environment can be interactive, and can have a high overhead.
	
	\item A \hypertarget{defn:production-environment}{\textbf{production environment}}, or use environment, refers to a setting in which software is no longer being modified by software developers and is being actively being used by users.
	Applications in production cannot have a high instrumentation/debugging overhead, as it is detrimental to the users.
	
	\item An \hypertarget{defn:error}{\textbf{error}}, also referred to as a defect or bug, is the deviation of system external state from correct service state.
	
	\item A \hypertarget{defn:fault}{\textbf{fault}} is the adjudged or hypothesized cause of an error.
	
	\item A \hypertarget{defn:failure}{\textbf{failure}} is an event that occurs when the delivered functionality deviates from correct functionality.
	A service fails either because it does not comply with the functional specification, or because this specification did not adequately describe the system function.
	
	\item \hypertarget{defn:devops}{\textbf{DevOps}} is a software development method that stresses communication, collaboration (information sharing and web service usage), integration, automation and measurement of cooperation between software developers and other information-technology (IT) professionals.
	DevOps acknowledges the interdependence of software development and IT operations.
	It aims to help an organization rapidly produce software products and services and to improve operations performance quality assurance.
	
	
	\item \hypertarget{defn:development-phase}{\textbf{Development/Operational Phase}} Development phase is the phase where the application is being developed.
	The process involves testing, and debugging and iterative development such as adding bug fixes etc. Operational phase is where the application is being operated and used by active users
	
	\item \hypertarget{defn:downstream-servers}{\textbf{Downstream Servers}} 
	For a given application or service, the downstream server is the server which sends it a request.
	
	\item \hypertarget{defn:upstream-servers}{\textbf{Upstream Servers}}
	For a given application or service, the upstream servers are servers which process it's requests and send it responses.
	
	
	\item \hypertarget{defn:production container}{\textbf{Production Container}}
	This is the container in which the original production service is hosted and where all incoming requests are routed.
	
	\item \hypertarget{defn:debug container}{\textbf{Debug Container}}
	This is a replica of the production container, where a copy of the production service is running. The debug container is used for debugging purposes, and provides the \livedebugging service.
	
	\item \hypertarget{defn:replica}{\textbf{Replica}}
	A replica is a clone of a container, with an exact clone of the file system and the processes running in the container. For the purpose of this thesis \debugcontainer and replica refer to the same thing.
	
	\item \hypertarget{defn:SOA}{\textbf{Service Oriented Applications}}
	Service oriented applications are applications which offer transactional services via network input, and provide responses on the network as well.
	
\end{itemize}

\section{Problem Statement}
\label{sec:introProblemStatement}

Despite advances in software engineering bugs in applications are inevitable.
The complexity of distributed and large scale applications, with an increased emphasis on shorter development cycles has made debugging more difficult.
The key challenge of debugging modern applications is twofold: firstly, the complexity due to a combination of distributed components interacting together, and secondly fast debugging of applications to assure a short-time-to debug.

We have observed that while several debugging techniques exist, most of them focus on localizing errors in the development phase.
Production level debugging techniques are ad-hoc in nature, and generally rely on unstructured logs printed as exceptions or transaction events using print outs from within the application.
While such logs are good, and can often give contextual information to the developer or the operator, they are meant to provide an indication to only expected errors.
Furthermore, they do not provide a systematic way to localize such bugs.

More systematic approaches such as record-and-replay systems offer a complete picture of the running production systems.
These tools capture the exact state, and execution of the system, and allow for it to be faithfully replayed offline.
This saves the debugger hours of effort in re-creating the bug, it's input and application state.
However, in order to capture such detailed information, there is a high performance penalty on the production systems.
This is often unacceptable in real-world scenarios, which is why such techniques have only found limited use.

We further observe that debugging is an iterative process.
While systematic approaches can provide a complete picture, developer insight is paramount.
The debugging process usually involves several iterations where the debugger uses clues present in error logs, system logs, execution traces etc. to understand and capture the source of the error.
This process can have an impact on real-world applications, hence traditionally the debugging and the production phase are kept completely separate.

Production level dynamic program instrumentation tools~\cite{dtrace,systemtap,winetw} enable application debugging, and live insights of the application. 
However, these are executed inline with the program execution, thereby incurring an overhead.
The perturbations and overhead because of the instrumentation could restrict the tools from being used in production environments. 
Thus we require a solution which allows operators/developers to observe, instrument, test or fix service oriented applications in parallel with the production. 
The techniques and mechanisms in this thesis will aim to provide a \livedebugging environment, which allows debuggers a free reign to debug, without impacting the user-facing application.

\section{Requirements}
\label{sec:introRequirements}

Our solution should meet the following requirements.

\begin{enumerate}
	\setlength{\itemsep}{0pt}
	\item \textbf{Real-Time Insights: }
	Observing application behavior as the bug presents itself will allow for a quick insight and shorter time to debug.
	Any solution should allow the debugger to capture system status as well as observe, whatever points he wishes in the execution flow.
	
	%\item \textbf{Understanding execution flows across distributed system}:
	%User requests for distributed applications go through several connected components.
	%This along with the concurrent nature of current applications makes tracking execution flow across machine extremely important.
	
	\item \textbf{Sanity and Correctness}:
	If the debugging is to be done in a running application with real users, it should be done without impacting the outcome of the program.
	The framework must ensure that any changes to the application's state or to the environment does not impact the user-facing production application.
	
	\item \textbf{Language/Application Agnostic}:
	The mechanisms presented should be applicable to any language, and any service oriented application (our scope is limited to SOA architectures).
	
	%\item \textbf{Allow for the easy creation/specification of tests}
	%The approach should allow software developers to easily create and specify the test cases, using familiar or easy-to-learn techniques.
	
	\item \textbf{Have negligible performance impact}
	The user of a system that is conducting tests on itself during execution should not observe any noticeable performance degradation. The tests must be unobtrusive to the end user, both in terms of functionality and any configuration or setup, in addition to performance.
	
	\item \textbf{No service interruption}: Since we are focusing our efforts on service oriented systems, any solution should ensure that there is not impact on the service, and the user facing service should not be interrupted.
	
\end{enumerate}


\section{Scope}
\label{sec:introScope}

Although we present a solution that is designed to be general purpose and applicable to a variety of applications, in this thesis we specifically limit our scope to the following:

\subsection{Service Oriented Applications}
\label{sec:introScopeSOA}

The traditional batch-processing single node applications are fast disappearing.
Modern day devices like computers, IOT's, mobile's and web-browsers rely on interactive and responsive applications, which provide a rich interface to it's end-users.
Behind the scenes of these applications are several \emph{SOA} applications working in concert to provide the final service. 
Such services include storage, compute, queuing, synchronization, application layer services.
One common aspect of all of these services is the fact that they get input from network sources.
Multiple services can be hosted on multiple machines(many-to-many deployment), and each of them communicates with the other as well as the user using the network.
The work presented in this thesis leverages duplication of network based input to generate a parallel debugging environment.
In this sense, the scope of the applications targeted in this thesis are limited to service oriented applications, which gather input through the network.

\subsection{Non-Crashing Bugs}
\label{sec:introScopeNonCrash}

In this thesis, we have primarily focused on continuous debugging in parallel with the production application.
We have looked at a variety of bugs - performance, resource leak, concurrency, semantic, configuration etc.
However, we also try to debug an active problem in the application. 

Hence, although a bug which immediately crashes, can still be investigated using \parikshan, it would not be an ideal use-case scenario.
On the other hand non-crashing bugs such as performance slow-downs, resource leaks which stay in the application long enough, fault tolerant bugs, which do not crash the entire system or similar non-crashing concurrency, semantic and configuration bugs, can be investigated in parallel to the original applications thereby reducing the investigation time, and the time to fix the bug.

\subsection{Native Applications}
\label{sec:introScopeNative}

One of the tools presented in this thesis is \iprobe - an intelligent hybrid instrumentation tool. 
\iprobe uses place-holders inserted at compile time in the binary, and leverages them to dynamically patch them at the run-time.
In it's current implementation \iprobe's techniques can be only applied on native applications.

Managed and run-time interpreted languages such as Java, and .NET can also theoretically have a similar approach built in, but that is out of the scope of this thesis.


\section{Proposed Approach}
\label{sec:introApproach}

Analyzing the executions of a buggy software program is essentially a data mining process.
Although several interesting methods have been developed to trace crashing bugs (such as memory violations and core dumps), it is still difficult to analyze non-crashing bugs.
Studies have shown that several bugs in large-scale systems lead to either a changed/inconsistent output, or impact the performance of the application.
Examples of this  are slow memory leaks, configuration, or performance bugs, which do not necessarily stop all services, but need to be fixed quickly so as to avoid degradation in the QoS.

Existing approaches towards debugging production bugs mostly rely on application logs, and transaction logs which are inserted within the application by the developer himself, to give an idea of the progress of the application, and to guide the debugger towards errors.
While these logs provide valuable contextual information, they can only be used for expected bug scenarios.
Furthermore, often they provide incomplete information, or are just triggered as exceptions without providing a complete trace.
Modern applications also contain a level of fault tolerance, which means that applications are likely to continue to spawn worker threads and provide service despite faults which happen at run-time.
This often means that the debugger loses the context of the application.

Other more systematic debugging techniques have been used in record-and-replay techniques which allow operators to faithfully capture the entire execution as well as the status of the operating system as well as the application.
This allows the debuggers to carefully debug the application offline and understand the root-cause of the bug.
However, an obvious disadvantage of such techniques is that the recording overhead can be relatively high, especially in unpredictable worst-case scenarios (for e.g. spikes in user requests etc.).
This makes the use of such techniques impractical for most real-world production systems.

Researchers have also studied non-systematic inference based techniques, which allow for lightweight tracing or capturing application logs in distributed applications, and then threading them together to form distributed execution flows.
These inference techniques~\cite{magpie,fmeter,vscope,clue,spectroscope} do not add much overhead to the production system, as they typically use production instrumentation tools, or existing application logs.
However, owing to the low amount of instrumentation and data captured, these tools focus on finding faults at higher granularity(module, library, component, node etc.) instead of the root-cause of the error at a code level (function, class, object etc.). 
Additionally most of these tools use logs from pre-instrumented binaries, thereby limiting them to expected bugs/error patterns.  

We propose a \textbf{ paradigm shift in debugging service oriented applications, with a focus on debugging applications running in the production environment}.
We call this technique \textbf{``live debugging''}: this technique will provide real-time insights into running systems, and allow developers to debug applications without fearing crashes in the production application.
We believe that this will in turn lead to much shorter time to bug resolution, hence improving application reliability, and reducing financial costs in case of errors.
In this thesis we present an \textbf{end-to-end work-flow of localizing production bugs, which includes a framework for live debugging, new live debugging techniques, and mechanisms to make applications live debugging friendly}.\\

%\noindent As a part of this thesis we answer the following key questions:
%\begin{itemize}
%	\item \textbf{How do we know which part of a distributed system to focus on for live debugging?}
%	\item \textbf{How can live debugging be safely applied, without impacting the application?}
%	\item \textbf{In what way can live debugging be used to localize bugs?}
%	\item \textbf{What are the various parameters we need to be aware of to localize these bugs?}
%\end{itemize}

\section{Hypothesis}
\label{sec:introHypothesis}

\noindent The principal hypothesis we test in this thesis is as follows:\\

\emph{For user-facing application where time-to-bug resolution is critical, network replay alone is enough to trigger most bugs in service-oriented applications, and sandboxed, on-the-fly debugging parallel to the production application can be enabled with minimal overhead on the production environment.}\\


In order to test this, we have developed the following technologies:
\begin{enumerate}
	\setlength{\itemsep}{0pt}
	\item A framework for sandboxed, online debugging of production bugs with no overhead (Parikshan)
	\item An intelligent compiler assisted dynamic instrumentation tool (iProbe)
	\item Applications of live on-the-fly debugging
\end{enumerate}


\begin{comment}
The main hypotheses investigated are as follows:
\begin{enumerate}
	
	\item \textbf{For user-facing application where time-to-bug resolution is critical, on-the-fly production debugging can be enabled without impacting the user by using live cloning}.
	That is, live migration and live cloning approaches can be applied to modern cloud based IAAS platforms to enable deeper inspection and debugging techniques with higher instrumentation overheads, without impacting user perceived performance.
	
	\item \textbf{Static compilation techniques combined with dynamic instrumentation can help greatly reduce instrumentation overhead, making it more acceptable for production systems}.
	Dynamic instrumentation can be complemented using compiler based techniques to make tracing production systems easier.
	We use custom compile time operations which insert \textbf{NOP} operation in the application binary, these can in turn be easily instrumented at run-time. 
	We further demonstrate the usage of our instrumentation mechanism by combining it with our \livedebugging framework as an effective means to do debugging.
	
	%\item \textbf{Adaptive instrumentation techniques can be used to keep instrumentation overhead within pre-designed budgets}. Here we pro-actively use queuing theory to set bounds for instrumentation in \parikshan, secondly we use re-active adaptive techniques using buffer usage as a trigger to adapt instrumentation.Adhering to this overhead allows us to avoid unnecessary buffer overflows in \parikshan, in turn extending the \debugwindow(explained later) for our framework.
	
	%\item \textbf{It is possible to use light-weight statistical mechanisms to localize bugs in a production system, while keeping a bounded instrumentation overhead}. Here we plan to use approaches similar to existing statistical debugging mechanisms to create effective \livedebugging techniques while adhering to a bounded overhead. We discuss how instrumentation points(called predicates) can be sampled at varying rates to have the maximum information gain, while not leading to buffer overflows in \parikshan.
	
\end{enumerate}

\end{comment}

\section{Assumptions}
\label{sec:introAssumptions}

The work presented in this thesis is designed so that it can be applied in the most generic cases. 
However, the implementation and some of the design motivation make some key assumptions which are presented in this section:

\subsection{Resource Availability}
One of the core insight driving our live debugging technology is the increasing availability of compute resources. 
With more and more applications being deployed on cloud infrastructure, in order to ease scaling out of resources and sharing of compute power across multiple services - The amount of computing power available is flexible and plentiful.
Several existing services like Amazon EC2~\cite{ec2} and Google Compute~\cite{gcompute} provide infrastructure-as-a-service and form the backbone of several well known cloud services. 

\parikshan assumes cheap and ample resource availability for most modern day services, and ease of scalability. We leverage this abundance of resources, to utilize unused resources for debugging purposes.
As mentioned earlier, \parikshan uses unused containers to run a replica of the original production service, solely for the purpose of debugging.
While it is difficult to quantify, we believe that the advantage of on-the-fly debugging and quick bug isolation outweighs the cost of these extra resources. 

\section{Outline}
\label{sec:introOutline}

The rest of this thesis is organized as follows:

\begin{itemize}
	\item Chapter~\ref{ch:parikshan} discusses the design and implementation of the \parikshan framework which enables live debugging.
	In this chapter we will first give a brief motivation, and discuss the overall design, and how our framework fits into service-oriented applications.
	We then go into a detailed explanation of the design of each of the components of network request duplication as well as our live cloning algorithm.
	We follow this up with implementation details, and evaluation scenarios using both simulation results and real-world experiments which show the performance of our framework. 
	
	\item Chapter~\ref{ch:NetworkReplaySurvey} we discuss case-studies involving 16 real-world bugs, from 5 well known service oriented application.
	We show how network input replay is enough to capture most real-world bugs (concurrency, performance, semantic, resource leak, and mis-configuration).
	In addition, to further help our claim, we did a survey of 217 real-world bugs which we manually classified and found were similar to the 16 bugs stated above.
	
	\item Chapter~\ref{ch:iprobe} introduces \iprobe a novel hybrid instrumentation technique. 	
	We first begin with an explanation of \iprobe's design, which is split in a two phase process - ColdPatching and HotPatching. This is explained in stateful diagrams to show how the code is modified at different states in the binary.
	We then show safety considerations of \iprobe and this is followed by an extended design which shows how \iprobe can be applied to applications without compile time modifications as well. 
	Next we compare \iprobe's approach with traditional trampoline executions. We then follow this with the implementation, and a short description of \textit{fperf} which is a application of \iprobe for hardware monitoring. 
	We follow this up with evaluation of \iprobe which shows \iprobe's overhead in cold-patching and hot-patching phase, and it's comparison with traditional tools.
	
	
	\item While the previous two chapters build the base for \livedebugging, Chapter~\ref{ch:activedebugging} discusses how these tools can be leveraged to do real-world debugging. In the first part of this chapter, we discuss several important advantages and limitations, which must be kept in mind when using \parikshan to debug applications. 
	Then we discusss existing debugging techniques which can be used in tandem with \livedebugging to provide a more effective means for localizing the bug.
	We also introduce a new technique called adaptive debugging. Adaptive debugging extends existing work on statistical debugging in  \parikshan to increase or decrease the degree of instrumentation in order to improve the statistical odds of localizing the bug.
	
	\item In chapter~\ref{ch:conclusions}, we conclude this thesis, highlighting the contributions of our techniques. 
	Additionally, this chapter also includes several future work possibilities that can arise from this thesis including some short-term future work and long-term possibilities.  
	
	
\end{itemize}
