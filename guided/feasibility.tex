\subsection{Feasibility}
\label{sec:guided_feasibility}

The current prototype for \activedebugging is in the early stages of development.
We have explored hotpatching in ~\cite{iprobe} and were able to successfully patch ELF binaries. 
Specifically our \iprobe tool enables dynamic instrumentation, and has been tested on several production level instrumentation techniques.
Other tools like Dyninst~\cite{dyninst}, and systemtap~\cite{systemtap} may also be used in tandem with ~\iprobe for our active debugging.

We have also explored previous approaches in the lab called INVITE (invivo testing~\cite{invivo}), which has looked into isolating test-cases in production environment. 
The core idea in invivo testing is to probabilistically trigger tests in the production environment. 
Furthermore, invivo ensures that these tests should not impact the production.
Other tools such as Java Assist, JAVA ASM also allow for binary modification and class re-loading in JAVA.
Applications like Tomcat, allow for dynamic class re-loading to speedup development. 
This inherently supports our design as we can design custom classes with forks which can do the test in the debug-container by generating two forks and killing them.
While complete isolation is not ensured in these approaches, forking execution does allow for process level isolation.
We have also explored file-system level isolation approaches using COW supported cloning mechanisms in BTRFS
The current approach will look at iteratively improving test-isolation guarantees in the debug-container, while the framework itself will provide complete isolation from production-container.
As part of the thesis we will integrate these approaches in the debug-container, and explore other applications for \livedebugging