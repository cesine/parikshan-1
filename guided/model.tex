\subsection{Statistical sampling of predicates with a bounded overhead}
\label{sec:guided_model}

In the previous chapter we discussed proposing a bounded overhead within which \parikshan can safely process logs, without requiring a re-cloning sync.
Here we use a statistical debugging approach which adheres to our bounded overhead.
We propose to assign to each predicate a weighted score which is associated with how frequently it should be monitored.
The sum of these weights will be equal to our bounded overhead.

In this work, we wish to establish a tradeoff between overhead of statistical debugging and the effectiveness of statistical debugging, at the same time maintaining a bounded overhead.
For instance, we can drop instrumentation of predicates which are executed more frequently to reduce the performance overhead in order to maintain our bounded overhead.
Program control and data flow inherently creates a hierarchical dependency between predicates, this can be leveraged in instrumentation to control the overhead, while at the same time provide some degree of clues to the debugger.

\subsubsection{Modeling the Target Application}

We define the program as a dynamic control-flow graph G \textless V,E \textgreater where V represent basic block entry points. 
The instrumentation granularity can be functional entries or conditionals which each should correspond to a basic block.
We call instrumentation points our predicates, where the predicate set \textless P \textgreater $\in$ \textless V \textgreater, 
i.e. each predicate can be an entry or exit point in the set of vertices's.


\subsubsection{Offline Profiling}

While not necessary, offline profiling of the target application can greatly assist in assigning weights to each predicate. 
Assuming that we have a representative input, the profile can be taken to show the time generally taken in each request, 
and every basic-block entry and exit point can be profiled to annotate time taken by each segment of the control path. 
While the processing time of each path can significantly differ depending on the input, 
we can get an approximate time taken by each section of the control flow graph for a large enough input set. 
This offline profiling helps us in assigning initial weights, and measuring the amount of slack available for each control-flow path.

\subsubsection{Assigning weight}

Weights can be defined to each predicate based on the following criteria:

\begin{itemize}
	
\item \textbf{Importance of the predicate itself}: 

The score for importance of a predicate can be defined similar to the Cooperative Bug Isolation Project (CBI) as shown in section ~\ref{sec:background-guided}. 

\begin{equation}
\label{eq:importance}
Importance = \frac{2}{\frac{1}{Sensitivity(p)}+\frac{1}{Increase(p)}}
\end{equation}


\item \textbf{Slack in context of the control flow path}: 

Model overall slack available based on offline instrumentation. 
 
\item \textbf{External factors}:

\end{itemize}

\textit{Amortizing Predicate Calculation}: Predicate scores are updated once every n runs

\subsubsection{Online Engine}

The job of the online engine is to decide whether or not a given predicate should be sampled. 
Inputs are 


\subsubsection{Cause Isolation}
\label{sec:cause_isolation}

Cause Isolation for bugs can be done in an adaptable manner. 

We assign weight W$_{i}$ to each predicate P$_{i}$ , where the weight denotes the sampling frequency.

\begin{equation}
    W_{i} = Probability\ that\ a\ predicate\ i\ is\ sampled
\end{equation}

\begin{equation}
    P_{i} = Instrumentation\ Overhead\ for\ Predicate\ i
\end{equation}

\begin{equation}
    \sum\limits_{i=1}^n W_{i}.P_{i} = Total\ Instrumentation\ Overhead\ Bound
\end{equation}

We have divided our approach in two halves, the first concentrates on statistical analysis for functional bugs.
Any application code can be modeled as a control flow graph 

\subsection{Motivating Example}
\label{sec:example}

We now discuss how live debugging can be used to find the cause of a bug using statistical debugging.
Take for example the code segment shown below:

\par
\begin{verbbox}[\footnotesize]
	void foo {
		if(x > num and f == NULL){
			x = 0;
			*f;
		}.
	}
\end{verbbox}
\fbox{\theverbbox}\par

In this code, the \texttt{*f} is clearly an erroneous call to this pointer, and will lead to an exception.
The function \texttt{foo}, can be called by various 
