

\section{Summary}
\label{sec:parikshanSummary}

\parikshan is a novel framework that uses redundant cloud resources to debug production SOA applications in real-time.
It can be combined with several existing bug diagnosis technique to localize errors.
Compared to existing monitoring solutions, which have focused on reducing instrumentation overhead, our tool is able to avoid any performance slowdown at all, at the same time potentially allow significant monitoring for the debugger.

We will make \parikshan available on GitHub for use by other researchers and practitioners.
For each of the 16 faults studied in our case study, we will also release a docker container (with README) that can be launched to trigger the bug.

%An extended version of the paper is present in CUCS Tech Reports~\cite{parikshanTR,parikshanQueue}, details about the project, and source code is available on github~\cite{github}. 
%Kaiser is funded in part by NSF CCF-1302269 and CCF-1161079.
%comment{
%We would like to acknowledge Qiang Xu, Abhishek Sharma, and Pallavi Joshi for their insight and feedback in designing \parikshan, and in the evaluation of this technology.
%The authors are affiliated with NEC Labs, Princeton, Google Inc, and Columbia University. 
