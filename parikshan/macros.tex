\newcommand{\iprobe}{\texttt{iProbe}\xspace}
\newcommand{\docker}{\emph{Docker}\xspace}
\newcommand{\parikshan}{\emph{Kensa}\xspace}
\newcommand{\livedebugging}{\textit{live debugging}\xspace}
%\newcommand{\debugcontainer}{\textit{debug container}\xspace}
\newcommand{\productioncontainer}{\textit{production container}\xspace}
\newcommand{\fio}{\textit{fio}\xspace}
\newcommand{\toolNameLang}{\emph{japanese}\xspace}
%\newcommand{\comment}[1]{}
\newtheorem{example}{Example}
\def\infinity{\rotatebox{90}{8}}

% Complex \xxx for making notes of things to do.  Use \xxx{...} for general
% notes, and \xxx[who]{...} if you want to blame someone in particular.
% Puts text in brackets and in bold font, and normally adds a marginpar
% with the text ``xxx'' so that it is easy to find.  On the other hand, if
% the comment is in a minipage, figure, or caption, the xxx goes in the text,
% because marginpars are not possible in these situations.
{\makeatletter
 \gdef\xxxmark{%
   \expandafter\ifx\csname @mpargs\endcsname\relax % in minipage?
     \expandafter\ifx\csname @captype\endcsname\relax % in figure/caption?
       \marginpar{\textcolor{red}{xxx~}}% not in a caption or minipage, can use marginpar
     \else
       \textcolor{red}{xxx~}% notice trailing space
     \fi
   \else
     \textcolor{red}{xxx~}% notice trailing space
   \fi}
 \gdef\xxx{\@ifnextchar[\xxx@lab\xxx@nolab}
 \long\gdef\xxx@lab[#1]#2{{\bf [\xxxmark \textcolor{red}{#2} ---{\sc #1}]}}
 \long\gdef\xxx@nolab#1{{\bf [\xxxmark \textcolor{red}{#1}]}}
 % This turns them off:
 \long\gdef\xxx@lab[#1]#2{}\long\gdef\xxx@nolab#1{}%
}

%Blue
{\makeatletter
 \gdef\yyymark{%
   \expandafter\ifx\csname @mpargs\endcsname\relax % in minipage?
     \expandafter\ifx\csname @captype\endcsname\relax % in figure/caption?
       \marginpar{\textcolor{blue}{yyy~}}% not in a caption or minipage, can use marginpar
     \else
       \textcolor{blue}{yyy~}% notice trailing space
     \fi
   \else
     \textcolor{blue}{yyy~}% notice trailing space
   \fi}
 \gdef\yyy{\@ifnextchar[\yyy@lab\yyy@nolab}
 \long\gdef\yyy@lab[#1]#2{{\bf [\yyymark \textcolor{blue}{#2} ---{\sc #1}]}}
 \long\gdef\yyy@nolab#1{{\bf [\yyymark \textcolor{blue}{#1}]}}
 % This turns them off:
 \long\gdef\yyy@lab[#1]#2{}\long\gdef\yyy@nolab#1{}%
}
