
\section{Applications of Live Debugging}
\label{sec:parikshanApplication}
%The concept of debugging typically encompasses more than just localizing the bug but also includes attempted fixes, tests of those fixes, and patching. 
%Here, we briefly discuss how existing techniques can be applied in \parikshan:

\noindent
\textbf{Statistical Testing:}
One well-known technique for debugging production applications is statistical testing. 
This is achieved by having predicate profiles from both successful and failing runs of a program and applying statistical techniques to pinpoint the cause of the failure.
The core advantage of statistical testing is that the sampling frequency of the instrumentation can be decreased to reduce the instrumentation overhead.
However, the instrumentation frequency for such testing to be successful needs to be statistically significant. 
Unfortunately, overhead concerns in the production environment limit the frequency of instrumentation.
In \parikshan, the buffer utilization can be used to control the frequency of such statistical instrumentation in the debug-container. 
This would allow the user to utilize the slack available in the debug-container for instrumentation to it's maximum, without leading to an overflow. 
Thereby improving the efficiency of statistical testing.

\noindent
\textbf{Record and Replay:}
Record and Replay techniques have been proposed to replay production site bugs. 
However, they are not yet used in practice as they can impose unacceptable overheads in the service processing time.
\parikshan replicas can be used to do recording at a much finer granularity (higher overhead), allowing for easy and fast replays offline.
Similar to existing mechanisms, the system can be replayed can then be used for offline debugging, without imposing any recording overhead to the production container.

\noindent
\textbf{Patch Testing:}
Bug fixes and patches to resolve errors, often need to undergo testing in the offline environment and are not guaranteed to perform correctly.
Patches can be made to the replica instead. 
The fix can be traced and observed if it is correctly working, before moving it to the production container. 
This is similar in nature to AB-Tesing, which is applied to find if a new fix is useful or works~\cite{abtesting}