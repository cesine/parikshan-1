
\subsection{Debug Window}
\label{sec:parikshanWindow}

\parikshan's asynchronous forwarder uses an internal buffer to ensure that incoming requests proceed directly to the production container without any latency, regardless of the speed at which the debug replica processes requests.
The incoming request rate to the buffer is dependent on the user, and is limited by how fast the production container manages the requests (i.e. the production container is the rate-limiter).
The outgoing rate from the buffer is dependent on how fast the debug-container processes the requests.
Instrumentation overhead in the debug-container can potentially cause an increase in the transaction processing times in the debug-container.
As the instrumentation overhead increases, the incoming rate of requests may eventually exceed the transaction processing rate in the debug container.
If the debug container does not catch up, this in turn can lead to a buffer overflow. We call the time period until buffer overflow happens the \emph{debug-window}.
This depends on the size of the buffer, the incoming request rate, and the overhead induced in the debug-container. 
For the duration of the debugging-window, we assume that the debug-container faithfully represents the production container. 
Once the buffer has overflown, the debug-container may be out of sync with the production container. 
At this stage, the production container needs to be re-cloned, so that the replica is back in sync with the production and the buffer can be discarded.
In case of frequent buffer-overflows, the buffer size needs to be increased or the instrumentation to be decreased in the replica, to allow for longer debug-windows.

The debug window size also depends on the application behavior, in particular how it launches TCP connections. 
\parikshan generates a pipe buffer for each TCP connect call, and the number of pipes are limited to the maximum number of connections allowed in the application.
Hence, buffer overflows happen only if the requests being sent in the same connection overflow the queue.
For webservers, and application servers, the debugging window size is generally not a problem, as each request is a new ``connection.''
This enables \parikshan to tolerate significant instrumentation overhead without a buffer overflow.
On the other hand, database and other session based services usually have small request sizes, but multiple requests can be sent in one session which is initiated by a user. 
In such cases, for a server receiving a heavy workload, the number of calls in a single session may eventually have a cumulative effect and cause overflows.

To further increase the \emph{debug window}, we propose load balancing debugging instrumentation overhead across multiple debug-containers, which can each get a duplicate copy of the incoming data. 
For instance, debug-container 1 could have 50\% of the instrumentation, and the rest on debug-container 2.
We believe such a strategy would significantly reduce the chance of a buffer overflow in cases where heavy instrumentation is needed.
Section~\ref{sec:parikshanTimewindowPerformance} explains in detail the behavior of the debug window, and how it is impacted by instrumentation.
\xxx{What happens when we exceed the debug window?}