

\subsection{Implementation}
\label{sec:parikshanImplementation}
%\xxx{I feel like maybe this section is describing the evaluation environment. In which case, we need to talk about that in such a context, and still say something about implementation}
%\xxx{Where is the architecture diagram? The system diagram? What did we build? What are the modules? How do they talk to each other? What are the system requirements? Does it *only* run on Nipun's specially configured laptop?}
%\xxx{We are talking about kernel version why? Does the implemetation have anything to do with hacking on the kernel? The distro? Why are we going on about it?}
The clone-manager and the live cloning utility are built on top of the user-space container virtualization software OpenVZ~\cite{openvz}.
\parikshan extends \emph{VZCTL} 4.8~\cite{vzctl} live migration facility~\cite{mirkin2008containers}, to provide support for online cloning.
To make \textbf{live cloning} easier and faster, we used OpenVZ's \textit{ploop} devices~\cite{ploop} as the container disk layout.
The network isolation for the production container was done using Linux network namespaces~\cite{netns} and NAT~\cite{nat}.
While \parikshan is based on light-weight containers, we believe that \parikshan can easily be applied to heavier-weight, traditional virtualization software where live migration has been further optimized~\cite{liveVMprinciples,trafficliveVM}.
%The reason for choosing a container based implementation was that containers take much less resources in comparison to traditional VM's.

The network proxy duplicator and the network aggregator was implemented in C/C++.
The forwarding in the proxy is done by forking off multiple processes each handling one send/or receive a connection in a loop from a source port to a destination port.
Data from processes handling communication with the production container, is transferred to those handling communication with the debug containers using \emph{Linux Pipes}~\cite{linuxpipes}.
Pipe buffer size is a configurable input based on user-specifications.